\documentclass[aspectratio=169,dvipsnames,c]{beamer}
\usepackage[T2A]{fontenc}
\usepackage[russian]{babel}
\usepackage{fontawesome5}
\usepackage{xeCJK}
\usepackage{tabularx}

\usepackage[absolute,overlay]{textpos}
\usepackage{pgf-pie}

% Подчёркивания
\usepackage{ulem}
\renewcommand{\ULdepth}{1.0pt}

\newcommand\Insertsubsection{}
\newcommand\Setsubsection[1]{
  \subsection{#1}
  \renewcommand\Insertsubsection{#1}
}

% Гиперссылки
\usepackage{hyperref}
\hypersetup{
  colorlinks=true,
  linkcolor=AccentColor,
  filecolor=AccentColor,
  urlcolor=AccentColor,
}

\newcommand\uhref[2]{{\color{AccentColor} \uline{\href{#1}{#2}}}}

\newcommand\ExampleIcon{{\color{AccentColor} \faFlask}}

\newcommand\ExampleNote{%
  \begin{textblock}{1}(15, 15)
    \ExampleIcon{}
  \end{textblock}
}

\newcommand\subtitlepage[2]{
  \subtitle{#1}
  \section{#1: #2}
  \begin{bg-frame}
    \vspace{1.75\baselineskip}
    {\Huge \insertsubtitle\\}
    \vspace{0.5\baselineskip}
    {\large #2}
  \end{bg-frame}
}

% Включение исходного кода.
\usepackage{minted}
\setminted{%
  breaklines,
  fontsize=\scriptsize,
  frame=leftline,
  linenos,
  showspaces,
  space=\char"22C5,
  spacecolor=Black!40,
  style=friendly, % Стили можно посмотреть командой pygmentize -L styles
}

\usepackage{smartdiagram}
\smartdiagramset{%
  module shape=rectangle,
  border color=Black!80,
  font=\scriptsize,
  text color=White,
  uniform color list=AccentColor for all items
}
\usesmartdiagramlibrary{additions}

% Отключает градиенты в рисунках smartdiagram, может повредить TikZ-графику
\tikzset{module/.append style={top color=\col,bottom color=\col}}
\tikzset{every shadow/.style={fill=none,shadow scale=0}}

\usepackage{shadowtext}
\shadowcolor{Black}
\shadowoffset{0.5pt}

\usetheme{custom}

\graphicspath{{bundled-img/}{img/}{icons/}}
\hypersetup{colorlinks=true}

\usepackage{relsize}

\newcommand\inlineicon[1]{
  {\larger[2] \color{AccentColor} \raisebox{-0.08\baselineskip}{#1}}
}

\newcommand\baselinespace{\vspace{\baselineskip}}

\newlength{\rulewidth}
\setlength{\rulewidth}{0.4pt}

\newcommand\roundpic[3][]{
  \begin{tikzpicture}[#1]
    \node[
      circle,
      minimum width = #2,
      path picture = {
          \node[] at (path picture bounding box.center){
            \includegraphics[width=#2]{#3}};
      }
    ]{};
  \end{tikzpicture}
}

% Requires to build twice to calculate position
\newcommand\overlaypic[3]{
  \begin{tikz}[remember picture, overlay]
    \node[anchor=#1] at (current page.#1) {\includegraphics[#2]{#3}};
  \end{tikz}
}

\newenvironment{Frame}[2][]
{%
  \subsection{#2}

  \begin{frame}[environment=Frame, {#1}]
    \frametitle{#2}
}
{%
  \end{frame}
}

\newenvironment{custom-bg-frame}[1]
  {%
    \begingroup
      \setbeamertemplate{background}{%
        \includegraphics[
          width=\paperwidth,
          height=\paperheight,
          keepaspectratio]{#1}
      }
      \color{White}
      \begin{frame}[c]
      \centering
  }
  {%
      \end{frame}
    \endgroup
  }

\newenvironment{bg-frame}
  { \begin{custom-bg-frame}{background.jpg} }
  { \end{custom-bg-frame} }

\newcommand\liveframe{
  \begin{bg-frame}
    \vspace{36pt}
    {\fontsize{60pt}{\baselineskip}\selectfont \bfseries LIVE}
  \end{bg-frame}
}

\newwrite\tempfile
\immediate\openout\tempfile=slidelist.txt

\newcounter{SlideNumber}
\addtobeamertemplate{frametitle}{}{%
  \stepcounter{SlideNumber}
  \immediate\write\tempfile{%
    \theSlideNumber\space\insertframetitle%
  }
}
