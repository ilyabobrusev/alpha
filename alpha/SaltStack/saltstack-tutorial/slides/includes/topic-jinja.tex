\subtitlepage{Jinja}{Шаблонизация формул}

\begin{Frame}[c]{Минутка истории}
  \overlaypic{south east}{width=75pt}{jinja-logo}

  \begin{centering}
    \huge
    Дзиндзя — синтоистский храм
    \footnote{Потому что temple созвучно с template}\\
  \end{centering}

  \vfill

  \begin{itemize}[<+-| @alert ->]
    \item Синтаксис шаблонизатора
      \href{https://www.djangoproject.com/}{Django}, больше возможностей
    \item На Python и для Python
    \item Поддерживает расширения
  \end{itemize}
\end{Frame}

\begin{Frame}[t]{Причины использовать шаблонизатор}
  \overlaypic{south east}{width=80pt}{thumbsup}
  \begin{itemize}[<+-| alert@ +>]
    \item Уменьшается дублирование кода
    \item Доступны grains, pillars, любые внешние данные
    \item Гибкость: условное включение кода
  \end{itemize}
\end{Frame}

\begin{Frame}[c]{Причины НЕ использовать шаблонизатор}
  \centering
  \overlaypic{south east}{width=80pt}{thumbsdown}

  \only<+->{%
    { \huge \bfseries СЛОЖНО\\ }
    Очень легко написать запутанный
    неподдерживаемый код\footnote<2>{Как и в случае с зависимостями}
  }

\end{Frame}

\Setsubsection{Синтаксис}
\begin{frame}[fragile]
  \frametitle{\Insertsubsection}
  \framesubtitle{Окружения}

  \begin{columns}
    \begin{column}{0.40\textwidth}
      \begin{itemize}[<+-| @alert ->]
        \item \texttt{\{\{ \}\}} --- выводимая инструкция \baselinespace{}
        \item \texttt{\{\% \%\}} --- блок [Jinja-кода] \baselinespace{}
        \item \texttt{\{{\# \#}\}} --- комментарий
      \end{itemize}
    \end{column}
    \begin{column}{0.55\textwidth}
      \begin{minted}[gobble=8, linenos=false, frame=lines]{jinja}
        <ul>
        
          {# Этот текст будет удалён шаблонизатором #}
          <li>{{ s }}</li>
        
        </ul>
      \end{minted}
    \end{column}
  \end{columns}

  \vfill

  \onslide<+->
\end{frame}

\begin{frame}{Синтаксис}
  \framesubtitle{Подрезка пробелов}
   \begin{itemize}[<+-| alert@ +>]
     \item[\faCut] Минусы убирают пробельные символы при подстановке
     \item[{\faArrowAltCircleLeft[regular]}] \texttt{\{\%- КОД\_JINJA \%\}} ---
       подрезаются символы слева
     \item[{\faArrowAltCircleRight[regular]}] \texttt{\{\% КОД\_JINJA -\%\}}
       --- подрезаются символы справа
     \item[\faParagraph] \texttt{'\textbackslash{}n'} --- это тоже пробельный
       символ
     \item[\faBan] Плюсы отменяют подрезку, указанную в конфиге шаблонизатора
   \end{itemize}
\end{frame}

\begin{frame}[fragile]
  \frametitle{\Insertsubsection}
  \framesubtitle{Инструкции}

  \setminted{linenos=false, frame=none}

  \begin{itemize}[<+-| @alert ->]
    \item Переменные\\
      \begin{minted}[gobble=8]{jinja}
        
      \end{minted}
    \item Цикл for
      \begin{minted}[gobble=8]{jinja}
        
        {{ i }}
        
      \end{minted}
    \item Условный оператор
      \begin{minted}[gobble=8]{jinja}
         Правда взаправду правда  Неправда 
      \end{minted}
    \item Фильтры и конвееры
      \begin{minted}[gobble=8]{jinja}
        {{ ['соль', 'мука', 'шаблонизатор'] | join(', ') | capitalize }}
      \end{minted}
  \end{itemize}
\end{frame}

\begin{frame}{Синтаксис}
  \framesubtitle{Макросы}

  \overlaypic{south east}{width=75pt}{books}

  \centering
  \inlineicon{\faFire}
  А ещё есть макросы --- это как процедуры\pause,\\
  но мы их пропустим.
\end{frame}

\Setsubsection{Рендеринг произвольного файла}
\begin{frame}[t, fragile]
  \ExampleNote{}

  \frametitle{\Insertsubsection}
  \begin{minted}[gobble=4]{py}
    import jinja2

    filename = 'template.txt'

    with open(filename, 'r') as f:
        template = jinja2.Template(f.read())

    print(template.render())
  \end{minted}
\end{frame}

\begin{Frame}[c]{Jinja и SaltStack}
  \begin{itemize}[<+-| @alert ->]
    \item[\faCube] Jinja подключается как text renderer \vfill
    \item[\faCube] Jinja используется по умолчанию \vfill
    \item[\faCube] Доступны специфичные для SaltStack словари (\texttt{grains},
      \texttt{pillars}) и специальный объект \texttt{salt}
  \end{itemize}
\end{Frame}

\begin{Frame}[t]{Jinja в формулах}
  \ExampleNote{}

  \inputminted[firstnumber=1, firstline=5,]{salt}{../srv/salt/jinja_sample.sls}
  $\nearrow$\footnote{Есть salt.states.win\_powercfg, но раньше он падал
  (\href{https://github.com/saltstack/salt/issues/57393}{\#57393})}

  \vfill{}

  \scriptsize
  \inlineicon{\faCubes} \pause{} \texttt{salt-call slsutil.renderer
  <путь-к-файлу>} чтобы отрендерить формулу.
\end{Frame}

\begin{frame}[t]{Jinja в формулах}
  \ExampleNote{}

  \framesubtitle{Pillar}
  \inputminted[firstnumber=1, firstline=5]{salt}{%
    ../srv/salt/jinja_pillar_sample.sls}

  \vfill{}

  \scriptsize
  \inlineicon{\faCubes} \texttt{salt-call saltutil.refresh\_pillar}
  чтобы синхронизировать пиллары.
\end{frame}

\begin{frame}[c]{Полезные ссылки}
  \setbeamertemplate{itemize items}[square]
  \begin{itemize}
    \item[\faLink] \uhref{https://jinja.palletsprojects.com/}{Документация Jinja}
    \item[\faLink] \uhref{%
        https://docs.saltproject.io/en/latest/topics/jinja/index.html
      }{Раздел документации SaltStack про Jinja}
  \end{itemize}
\end{frame}
